
\begin{frame}
		% Big data has different meanings to different people.  Before any
		% discussion about solving problems reached with "Big Data", we need to
		% define what we are talking.  
		
		Data sets that require unique solutions to analyze, share or store are
		consider Big Data

		% What does this mean?  In practice, we experience this when we run a
		% data set through a known working workflow or utility, and it fails to
		% work.  It either takes too long to complete (months or years) or
		% crashes due to RAM consumption (computer freezes, maybe?).
		% Additionally, some data sets are so large they can't even fit on the
		% storage of your workstation or lab server.  While there are a lot of
		% different challenges that can be presented with Big Data, including
		% generating/collecting, annotating, appropriate statistical methods,
		% etc... Here we are going to specifically deal with 
\end{frame}



